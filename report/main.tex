%--------------------
% Packages
% -------------------
\documentclass[11pt,a4paper]{article}
\usepackage[utf8x]{inputenc}
\usepackage[T1]{fontenc}
%\usepackage{gentium}
\usepackage{mathptmx} % Use Times Font


\usepackage[pdftex]{graphicx} % Required for including pictures
\graphicspath{ {images/} }
\usepackage[swedish]{babel} % Swedish translations
\usepackage[pdftex,linkcolor=black,pdfborder={0 0 0}]{hyperref} % Format links for pdf
\usepackage{calc} % To reset the counter in the document after title page
\usepackage{enumitem} % Includes lists

\frenchspacing % No double spacing between sentences
\linespread{1.2} % Set linespace
\usepackage[a4paper, lmargin=0.1666\paperwidth, rmargin=0.1666\paperwidth, tmargin=0.1111\paperheight, bmargin=0.1111\paperheight]{geometry} %margins
%\usepackage{parskip}

\usepackage[all]{nowidow} % Tries to remove widows
\usepackage[protrusion=true,expansion=true]{microtype} % Improves typography, load after fontpackage is selected



%-----------------------
% Set pdf information and add title, fill in the fields
%-----------------------
\hypersetup{ 	
pdfsubject = {Machine Learning 2 - Final Assignment},
pdftitle = {Machine Learning 2 - Final Assignment},
pdfauthor = {Frans de Boer}
}

\author{
    Frans de Boer \\
    frans.deboe \\
    5661439 \\
}
\title{Machine learning 2 - Final Assignment}


%-----------------------
% Begin document
%-----------------------
\begin{document} %All text i dokumentet hamnar mellan dessa taggar, allt ovanför är formatering av dokumentet


\section{PAC Learning}

\subsection{Extend the proof that was given in the slides for the PAC-learnability of hyper-rectangles:
show that axis-aligned hyper-rectangles in n-dimensional feature spaces (n > 2) are PAC
learnable.}

First note: This is a consistent learner so yes it is PAC learnable. Probably not the proof you wanted to see though (:

% \begin{figure}[h]
%     \centering
%     \includegraphics[\linewidth]{PAC-2n.png}
%     \caption{PAC Learnable hyper-rectangle with $n=2$}
%     \Description{An example of the weighted maximum cut problem containing 5 nodes}
%     \label{fig:pac-2n}
%   \end{figure}
  


\end{document}